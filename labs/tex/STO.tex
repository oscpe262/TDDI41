%% Doc Props
\documentclass[11pt,a4paper]{report}
\usepackage[utf8]{inputenc}
\usepackage[english]{babel}

% Packages
\usepackage{
  amsmath,
  titlesec,
  tikz,
  pifont,
  textcomp,
  fixltx2e,
  xcolor,
  fullpage,
  graphicx,
  afterpage,
  lastpage,
  float,
  fancyhdr,
  fancyvrb,
  parskip,
  xfrac,
  gensymb,
  framed,
  caption,
  subcaption,
  mathtools,
  pdfpages,
  etoolbox,
  enumitem,
  hyperref
}
\usepackage[nottoc,section,numbib]{tocbibind}
%\usepackage[europeancurrents,europeanvoltages,europeanresistors,europeaninductors,smartlabels]{circuitikz}
%\usepackage{pgfplotstable}
%\pgfplotsset{compat=1.12}
\usetikzlibrary{calc,matrix,positioning,arrows,shapes,trees,plotmarks,decorations.markings,shapes.geometric}
\usepackage[font={small,it}]{caption} %% Italics in captions
\tikzstyle{vecArrow} = [thick, decoration={markings,mark=at position 1 with {\arrow[semithick, fill=green]{triangle 60}}}, double distance=1.4pt, shorten >= 5.5pt, preaction = {decorate}, postaction = {draw,line width=1.4pt, green,shorten >= 4.5pt}]
\tikzstyle{innerWhite} = [semithick, green,line width=1.4pt, shorten >= 4.5pt]

\tikzset{
  treenode/.style = {align=center, inner sep=0pt, text centered,font=\sffamily},
  arnn/.style = {treenode, circle, white, font=\sffamily\bfseries, draw=black,fill=black, text width=1.5em},
  arnr/.style = {treenode, circle, red, draw=red, text width=1.5em, very thick},
  arnx/.style = {treenode, rectangle, draw=black,  minimum width=0.5em, minimum height=0.5em}
}

% PDF props
\hypersetup {
  bookmarks=true,              % show bookmarks bar?
  unicode=true,                % non-Latin characters in Acrobat’s bookmarks
  pdftoolbar=true,             % show Acrobat’s toolbar?
  pdfmenubar=true,             % show Acrobat’s menu?
  pdffitwindow=false,          % window fit to page when opened
  pdfstartview={FitH},         % fits the width of the page to the window
  pdftitle={TDDI41},                 % title
  pdfauthor={Oscar Petersson}, % author
  pdfsubject={},          % subject of the document
  pdfkeywords={}          % list of keywords
  pdfnewwindow=true,           % links in new window
  hidelinks,                   % hide links (removing color and border)
  linktoc=all,                 % parts of TOC made into links
}
\urlstyle{same}

% More Props
\graphicspath{{./fig/}}

% Comfy Command Redefinitions

%% First Page Information
\author{Oscar Petersson, Matteus Laurent\\oscpe262, matla782}
\title{TDDI41 Lab report}
\date{\today}
\newcommand{\course}{System Administration}
\newcommand{\coursenumber}{TDDI41, Linköpings universitet}
\newcommand{\programme}{Högskoleingenjörsutbildning i datateknik, 180 hp}
\newcommand{\examiner}{Jon Dybeck}
\newcommand{\institution}{IDA}
\newcommand{\reptype}{Laboration report}

%%% Document %%%
\begin{document}

\setcounter{secnumdepth}{-1}
\setcounter{tocdepth}{-1}

% Custom titlepage, empty backside, do not use with \maketitle
\pagestyle{empty}
  \makeatletter\begin{center}~\vfill
  \Large{\@author}\vskip .3cm
  \textbf{\LARGE{\@title}}\vskip .2cm
  \large{\programme}
  \vfill
\end{center}
\reptype{} - \@date\hfill Assistant:\\
\textbf{\course}\hfill\examiner\\
\coursenumber\hfill\institution
\makeatother
\newpage
\pagestyle{plain}
%% Table of Contents
%\addtocontents{toc}{\protect\hypertarget{toc}{}}
%\tableofcontents\label{sec:toc}
%\addtocontents{toc}{\protect\thispagestyle{empty}}
%\cleardoublepage
% Call First File Branch
%\pagestyle{plain}
\setcounter{page}{1}

\section{UML}
\subsection{3-2}
64M for lab1, lab2 server has 128M

\subsection{3-3}
A ``wrapper for terminal sessions'', or rather, an emulated terminal device. It allows the terminal session to be kept running (and resumed later) even when the terminal client itself is closed.

\subsection{6-1}
\verb=$ scp root@10.17.1.212:/etc/network/interfaces <wherever-we-want-to-put-it>=

\subsection{6-2}
\verb=$ scp -r root@...:/etc/default <destination>=

\input{../report.input.tex}
\chapter{LXB}
\section{Exercise 3: Navigating the Unix manual}
\subsection{3-1 Answer the following questions concerning sections of the Unix Manual. Most of the information you need can be found in the man page for the man command (you will need to think a little too).}
\subsubsection{a) Which are the nine sections of the Unix manual.}
\begin{tabular}{ll}
1&   Executable programs or shell commands\\
2&   System calls (functions provided by the kernel)\\
3&   Library calls (functions within program libraries)\\
4&   Special files (usually found in /dev)\\
5&   File formats and conventions eg /etc/passwd\\
6&   Games\\
7&   Miscellaneous (including macro packages and conventions), e.g. man(7), groff(7)\\
8&   System administration commands (usually only for root)\\
9&   Kernel routines
\end{tabular}

\subsubsection{b) Which section of the manual contains user commands such as cat and ls.}
Section 1

\subsubsection{c) Which section documents file formats, such as configuration files.}
Section 5

\subsubsection{d) Which section contains system administration commands, such as halt.}
Section 8
\subsection{4-1 The SYNOPSIS section briefly lists how a command is invoked. Read the man page for man and explain what the following command synopsis texts mean. Explain how each command may or must be invoked; you don't need to know what the commands actually do.}
a) \verb=mkpasswd PASSWORD SALT=
\begin{verbatim}command argument argument\end{verbatim}

b) \verb=uniq [OPTION]... [INPUT [OUTPUT]]=

\begin{verbatim}command [runtime options goes here]...
[optional input stream[optional output stream]]\end{verbatim}


c) \verb=gzip [-acdfhlLnNrtvV19 ] [-S suffix] [ name ... ]=

\begin{verbatim}
command [-available runtime flags w/o further arguments]
[-S flag needs argument (filename suffix)] [optional output name]
\end{verbatim}

d)

\begin{verbatim}chcon [OPTION]... CONTEXT FILE...
chcon [OPTION]... [-u USER] [-r ROLE] FILE...
chcon [OPTION]... --reference=RFILE FILE...\end{verbatim}

Three possible invokation syntaxes:
\begin{verbatim}
     command [runtime option] argument argument
     command [runtime option] [-u-flag argument] [-r-flag argument] argument
     command [runtime option] --reference-flag=argument argument
\end{verbatim}


\subsection{4-2 Read the man page for man, as well as some other man pages (e.g. for ls, uniq, chmod, and adduser) and answer the following questions.}
\subsubsection{a) What do you usually find in the DESCRIPTION section.}
A brief briefing of what the entry is about, i.e. for an executable - what the executable' function is, and a description of flags etc.

\subsubsection{b) Which section(s) usually document, in detail, what each command-line option does.}
Section 1

\subsubsection{c) Let's say you're reading the man page for a command and the information you are looking for isn't there. In which part can you find references to other man pages that might contain what you are looking for (have a look at the man page for reboot, imagining that you are trying to find a command that will turn the computer off, for an example).}
Under \textbf{SEE ALSO}

\subsubsection{d) In which section do you find information about which configuration files a program uses?}
Section 1, usually a subsection


\subsection{4-3 Use the man page for man to answer the following questions.}

\subsubsection{a) Sometimes there are several man pages (located in different sections) for the same keyword. Which command-line option to man can you use to display all of them.}
\verb=man -a <search term>=

\subsubsection{b) Sometimes you don't know the name of a command you are looking for, but can guess at a keyword related to the command. Which command-line option can you use to have man list all pages related to a specific keyword.}
\verb=man -k <keyword>=

\subsection{4-4 Display the man page for the ls command.}
\subsubsection{a) What does the ls command do.}
It lists info about files in a given search path.

\subsubsection{b) What option to ls shows information about file sizes, owner, group, permissions and so forth.}
\verb=-l=

\subsubsection{c) What does the -R option to ls do? (Don't forget to try it.) Report: Brief written answers to the questions above.}
Adds recursive directory search (lists subdirectories and their content as well).

\section{Exercise 5: Absolute and relative path names}
\subsection{5-1 In the example above name at least one relative path name indicating ssh if:}
\subsubsection{a) The current working directory is /usr/bin.}
\verb=./ssh=

\verb=../bin/ssh =

\subsubsection{b) The current working directory is /usr/local/bin.}
\verb=../../bin/ssh=

\section{Exercise 6: Long format chmod}
\subsection{6-1 It is possible to set individual permissions for user, group and others using chmod. Review the documentation and answer the following questions.}

\subsubsection{a) How can you set the permission string to user read/write, group read, others read using chmod in long format.}
\verb=# chmod a+rw-x,go-w <file>=

\subsubsection{b) How can you revoke group write permissions on a file without changing any other permissions.}
\verb=# chmod g-w <file>=

\subsubsection{c) How can you grant user and group execute permissions without changing any other permissions.}
\verb=# chmod ug+x foo=



\section{Exercise 7: Numeric file modes}
\subsection{7-1 What do the following numeric file modes represent.}

\subsubsection{a) 666}
\verb=rw-rw-rw-=

\subsubsection{b) 770}
\verb=rwxrwx---=

\subsubsection{c) 640}
\verb=rw-r-----=

\subsubsection{d) 444}
\verb=r--r--r--=

\subsection{7-2 What command-line argument to chmod allows you to alter the permissions of an entire directory tree.}
\verb=-R, --recursive=

\subsection{7-3 What does execute (x) permission mean on directories.}
Allows entering the directory and listing of its content.

\subsection{7-4 A user wants to set the permissions of a directory tree rooted in dir so that the user and group can list, read and write (but not execute) files, but nobody else has any access. Which of the following commands is most appropriate? Why?}

\subsubsection{a) chmod -R 660 dir}

\subsubsection{b) chmod -R 770 dir}

\subsubsection{c) chmod -R u+rw,g+rw,o-rwx dir}

\emph{(c)} is, arguably, the least wrong one, as it doesn't change possibly existing executive permissions for \verb=[ug]=. \emph{(a)} would remove any executive permissions on all files (and directories, making them inaccessible) for anyone, and \emph{(b)} would add executive permissions for \verb=[ug]= to all files and directories in the tree.

The prerequisits for \emph{(c)} would be that the executive permissions on the files already are correct.

The \emph{truly} correct answer would on the other hand be that a \emph{user} should not have the right to chmod recursively anywhere in the tree apart from their user-specific directories, and those should not be writable to anyone else.


\section{Exercise 8: Owner and group manipulation}
\subsection{8-1 How can you change the owner and group of an entire directory tree (a directory, its subdirectories and all the files they contain) with a single command.}
\verb=# chown user:group -R <path root>=

\section{Exercise 9: File manipulation commands}
\subsection{9-1 What does cd .. do.}
Changes current work directory to the parent directory.

\subsection{9-2 What does cd ../.. do.}
Changes current work directory to the parent directory of the parent directory.

\subsection{9-3 What information about a file is shown by ls -laF?}
Current directory content in the format:\\
\verb=filemode <> user group filesize(bytes) modification_time filename[entry type indicator]=

\subsection{9-4 In the following output from}
\begin{verbatim}
  ls -laFd dir dsp:
  drwxr-xr-x 22 dave staff 4096   Jan 12 2001 dir/
  crw-rw----  1 root audio 14, 3  Jan 22 2001 dspp
\end{verbatim}
\subsection{Explain the following:}
\subsubsection{a) What does the "c" at the beginning of the second line mean.}
Character special file, e.g. pipes, ports, devices etc ...

\subsubsection{b) What do "dave" and "staff" mean on the first line, and "root" and "audio" on the second.}
User permissions for the files are set to the users "dave/root" and groups "root/audio" respectively.

\subsubsection{c) Which users may write to the file dspp.}
The user "root" and any users in the group "audio".

\subsection{9-5 If you have two files, a and b, and you issue the command mv a b, what happens? Is there an option to mv that will issue a warning in this situation.}
rename file 'a' to 'b'. \verb=mv -i= will interactively ask for permission to overwrite.

\subsection{9-6 What option(s) to cp allows you to copy the contents of /dir1 to /dir2, preserving modification times, ownership and permissions of all files.}
\verb=$ cp -a /dir1/*.* /dir2/=

\subsection{9-7 How do you make the file secret readable and writable by root, readable by the group wheel and completely inaccessible to everybody else.}
\verb=$ chown root:wheel secret && chmod 640 secret=

\section{Exercise 10: Shell init files}
\subsection{10-2 What init files does your shell use, and when are they used? (Hint: your shell has a man page, and somewhere near the end there is a section that lists which files it uses).}

\begin{tabular}{ll}
\verb=/bin/bash= &
	The bash executable\\
\verb=/etc/profile= &
	The systemwide initialization file, executed for login shells\\
\verb=/etc/bash.bashrc= &
	The systemwide per-interactive-shell startup file\\
\verb=/etc/bash.bash.logout= &
	The systemwide login shell cleanup file, executed when a login shell exits\\
\verb=~/.bash_profile= &
	The personal initialization file, executed for login shells\\
\verb=~/.bashrc= &
	The individual per-interactive-shell startup file\\
\verb=~/.bash_logout= &
	The individual login shell cleanup file, executed when a login shell exits\\
\verb=~/.inputrc=&
	Individual readline initialization file
\end{tabular}

\section{Exercise 11: Manipulating environment variables}
\subsection{11-1 Use the env command to display all environment variables. What is PATH set to (you might want to use grep to find it)? What is this variable used for (the man pages for your shell might be helpful in answering this question).}
\verb+PATH=/bin:/sbin:/usr/local/sbin:/usr/local/bin:/usr/bin:/usr/bin/site_perl:\+\\
\verb+/usr/bin/vendor_perl:/usr/bin/core_perl+

\verb=$PATH= is a list of paths from where the shell will recognize executables. Executables in \verb=$PATH= can be executed from any search path without writing the full path to the executable.

\subsection{11-2 Use echo to display the value of HOME. What does the HOME variable normally contain.}
\verb=$HOME= is the user "default" directory. Here the users document, user-specific configurations etc. are stored.

\subsection{11-3 Prepend /data/kurs/adit/bin:/data/kurs/TDDI09/bin to the variable PATH. The easiest way to accomplish this is to use variable expansion in the right-hand side of the assignment.}
\begin{verbatim}$ PATH='/data/kurs/adit/bin:/data/kurs/TDDI41/bin:'${PATH}\end{verbatim}

\section{Exercise 12: Redirecting output}
\subsection{12-1 Where will stdout and stderr be redirected in the following examples?}
\verb=a) command >file1 b) command 2>&1 >file1 c) command >file1 2>&1=
\subsection{When answering these, remember that the order of redirections matters.}
\begin{tabular}{l|ll||l|ll}
stdout: &&&	stderr:&&\\\hline
 &  file1 &console&	  & file1& console\\\hline
a&  y&	 n &	  a&  n&     y\\
b&  y&   n &		b&  n&     y\\
c&  y&	 n &		c&  y&	   n
\end{tabular}

\section{Exercise 13: Pipelines}
\subsection{13-1 What do the following commands do?}
\begin{verbatim}
a) ls | grep -i doc
b) command 2>&1 | grep -i fail
c) command 2>&1 >/dev/null | grep -i fail
\end{verbatim}
a) List directory content, select only the lines which contains the ordered character set "doc" ignoring upper/lower case.

b) Redirects stderr output of \verb=command= to stdout which is then parsed for "fail" (ignoring u/l case)

c) As b, but will also put the stdout output into The Great Void.
\subsection{13-2 Write command lines to perform the following tasks.}
\subsubsection{a) Output a recursive listing (using ls) of your home directory, including invisible files, to the file /tmp/HOMEFILES.}
\verb=$ ls -aR ~ > /tmp/HOMEFILES=

\subsubsection{b) Find any files (using find) on the system that are world-writable (i.e. the write permission for "others" is set). Error messages should be discarded (redirected to /dev/null). This command is actually useful for auditing the security of a system - world-writable files can be security risks.}
\verb?# find / -perm -o=w -type f 2>/dev/null?


\subsection{Exercise 14: Processes and jobs}
\subsection{14-1 Create a long running process by typing ping 127.0.0.1. Suspend it with vZ and bring it to the foreground with fg. Terminate it with vC.}

\subsection{14-2 Create a long running process in the background by typing} \verb=ping 127.0.0.1 >/dev/null&=.
\subsection{Find out its process id using ps and kill it using kill.}
\begin{verbatim}
$ ping 127.0.0.1 >/dev/null&
kill $(ps -Ao pid=,comm --no-headers | grep ping | grep -o -E '[0-9]+')
\end{verbatim}

(Or, just "ps", look for ping, send \verb=kill <found PID>=), but that's a rather boring way, isn't it? ;)

\subsection{14-3 What does the command kill -9 pid do, where pid is the number of a process? What does kill -9 -1 do? Read the documentation to figure the last one out as it is a somewhat dangerous command.}
\verb=$ kill -9 <pid>=

This line sends \textbf{SIGKILL} to the process with given PID. This can not be caught or ignored by the process.

\verb=$ kill -9 -1=

This line sends \textbf{SIGKILL} to all processes with a PID larger than 1, i.e. every process apart from the init process started by the boot manager.


\subsection{14-4 Create a long running process in the background. Kill it using pkill. The pkill command is very useful when you need to kill several processes that share some attribute (such as a command name).}
\verb=$ pkill ping=


\section{Exercise 16: Using the pager less}
\subsection{16-1 What keystroke in less moves to the beginning of the file.}
\subsection{16-2 What keystroke in less moves to the end of the file.}
\subsection{16-3 What would you type in less to start searching for "option".}
\subsection{16-4 What would you type in less to move to the next match for "option".}
BOF: g\\
EOF: G\\
forward search: /option\\
backward search: ?option

\subsection{16-5 Locate the package documentation for the ssh package and answer the following questions by reading the README.Debian.gz file (hint: remembering the answers to these questions may be useful in the project).}

\subsubsection{a) What is the default setting for ForwardX11.}
For security reasons, \verb=X11Forwarding= is disabled by default, this goes for Debian's \verb=openssh-server= package as well.

\subsection{b) If you want X11 forwarding to work, what other package(s) need to be installed on the server.}
\verb=xauth= at a bare minimum, plus of course any related packages for whatever functionality desired.



\subsection{Exercise 17: Using non-interactive text editors (this exercise is optional)}
\subsection{17-1 Use sed to change all occurrences of "/bin/tcsh" to "/bin/sh" in /etc/passwd (output to a different file). This exercise is optional.}
\verb=# sed -i "s|bin/tcsh|bin/sh|g" < /etc/passwd > foo=

\subsection{17-2 Examine the files shadow and passwd in the directory /data/kurs/TDDI09/labs/lxb. Use paste and awk to output a file where each line consists of column one from passwd and column two from the corresponding line in shadow. The printf function in awk is helpful here. This exercise is optional since it goes beyond the basics.}
\begin{verbatim}$ paste <(awk -F ":" '{print $2}' passwd ) <(awk -F ":" '{print $3}' shadow)\end{verbatim}

\section{Exercise 18: Log files}
\subsection{18-1 What does tail -f /var/log/syslog do.}
Prints the last 10 lines of \verb=/var/log/syslog= and keeps parsing.

\subsection{18-2 If you want to extract the last ten lines in /var/log/syslog that are related to the service cron, what command would you use? (Hint: the grep command can search for matching lines in a file).}
\verb=$ cat /var/log/syslog | grep cron=

%\section{Part 5 Errata:}
% The default configuration for most contemporary distributions is systemd, regardless of what one thinks of that. Sure, sysvinit might be a reasonable (at least you can reason wether it is or not ...) choice regarding this course, but that does not change the fact that the major distributions likely to be found in a professional environment (Debian, Fedora, SUSE, Red Hat, \*buntu, Mint)- except for legacy systems - are not running sysvinit.
% Errata-errata: Well, to be fair, likely to be found is old legacy systems, and Ubuntu is running upstart etc. So, never mind this crap.

\section{Exercise 19: Booting Linux}
\subsection{19-1 What services are started when your system boots into run level 2 (i.e. not only services exclusive to run level 2, but all services started as the computer boots into run level 2).}
bootlogs, syslog, ssh, cron, rmnologin, stopbootlogs

\subsection{19-2 If you wanted to restart the ssh process, what command would you use.}
\begin{verbatim}
$ systemctl restart sshd.service
-bash: systemctl: command not found
\end{verbatim}

Oh right, old habits die hard ... :p

\verb=$ /etc/init.d/ssh restart=


\section{Exercise 20: Mostly hard stuff}
\subsection{20-1 Where will stdout and stderr be redirected in the following examples:}
\verb=a) command 2>&1 >file1 b) command >file1 2>&1=
\emph{see 12-1 b/c}

\subsection{20-2 Using only ps, grep and kill, write a command that kills any process with "linux" in the name. Note that you must avoid killing the grep process itself (this can be done by crafting the regexp appropriately)! This command line can be used to rapidly terminate any UML running on the system.}
\begin{verbatim}$ kill $(ps -Ao pid,comm --no-headers | grep ping | grep -o -E '[0-9]+')\end{verbatim}

\subsection{20-3 Use your shell's alias or function features to create a shell command named gzless that decompresses a gzip-compressed file and pipes it into less.}
\begin{verbatim}
# ~/.aliases
gzless() {
  gunzip -c $1 | less
}
\end{verbatim}

\subsection{20-4 I/O redirection in the shell is somewhat limited. In particular, you can't handle programs that write directly to the terminal device (i.e. use a pseudo-terminal) rather than use the file descriptors. For situations like that, there's socat. Since you're an advanced user, you really should learn about socat.}
\subsubsection{a) Use socat so you can log in and execute commands on a host non-interactively using the password authentication method. Essentially, you should be able to do the equivalent of}
\verb= cat commandfile | ssh user@remote.host =
\subsubsection{, where commandfile contains the stuff you would normally type interactively.}
\begin{verbatim}
$ file=$(cat commandfile) && \
  (sleep 3; echo PASSWD; sleep 1; echo "$file"; sleep 1) |\
  socat - EXEC:'ssh -l USER remote.ip.addr',pty,setsid,ctty
\end{verbatim}

\subsubsection{b) Do something else with socat that's neat. Come up with something on your own, perhaps using the network.}

\subsection{20-5 Write a command that renames a bunch of files to be all lowercase (e.g. BOFH.GIF is renamed to bofh.gif). You do not have to worry about spaces (but you should, just on general principle). The tr command may be useful here.}
\begin{verbatim}
#!/bin/bash

for OLD in `find $1 -depth`
do
   NEW=`dirname "${OLD}"`/`basename "${OLD}" | tr '[A-Z ]' '[a-z ]'`
   if [ "${OLD}" != "${NEW}" ]
   then
	[ ! -e "${NEW}" ] && mv -T "${OLD}" "${NEW}" \
	|| echo "Could not rename ${OLD}. ${NEW} already exists."
   fi
done
\end{verbatim}

\subsection{20-6 Examine the files /data/kurs/TDDI09/labs/lxb/passwd and shadow (same directory). Use paste and awk to output a file where each line consists of column one from passwd and column two from the corresponding line in shadow. The printf function in awk may be helpful here.}
\emph{see 17-2}

\subsection{20-7 Sometimes you need to change the IP address of a computer. Assuming that all the files that need to be changed are somewhere in /etc (or a subdirectory thereof), what command will list all relevant files and not print any error messages.}
\verb=# find ./ -type f | xargs grep "$oldip" 2>/dev/null=
...where \verb=$oldip= is the IP-address we are looking for, of course.

\subsection{20-8 Write a command line that changes the IP address stored in the variable (shell or environment) OLD to the IP address stored in the variable NEW in all (regular) files in /etc or its subdirectories. The commands find, xargs and sed may be useful here.}
\verb=# find /etc -type f -exec sed -i "s|$OLD|$NEW|g" {} \;=

\subsection{20-9 Write a command that uses ssh to log in to all computers w0001 through w2000 and runs the w command on each one. You might want to read about arithmetic substitution in bash for this one. How could you avoid having ssh ask for a password for each computer without setting an empty password or resorting to host-based authentication.}

Avoiding password login would be done through pubkey-based access:

\begin{verbatim}
#/etc/ssh/sshd_config
 PasswordAuthentication no
 PermitEmptyPasswords no
 [...]
#EOF
\end{verbatim}

Generating and importing keys ...
\verb=ssh-keygen=

We could use \verb=sshpass -p <PASSWD> ssh-copy-id -i ~/.ssh/<id-file> <user>@<host>= to avoid manually entering the passwords, at least if they are identical on all hosts...

Once sorted (phew) we'll simply run:

\begin{verbatim}
$ for client in `seq 0001 2000`
  do
    ssh user@$client <<FOOXECUTETHIS
    w
    FOOXECUTETHIS
  done
\end{verbatim}
\section{Exercise 21: Quite hard stuff (optional)}
\subsection{21-1 Write a command that lists, for the last ten unique users to log in to the system, the last time they logged in using ssh (users can be found using last, and ssh logins in /var/log/auth.log, which is typically only readable by root).}

\subsection{21-2 Explain the following fairly contrived code, in particular all the I/O redirections.}
The code starts out with directing two file descriptors, one for input and one for output (l.2). We look for .bak-files throughout the file system and print their path on a line each (l.3). We read these lines (l.5), ask (l.6) the user to confirm that we should log the entry to \verb=/tmp/RECORD (l.2)=. If (l.8) the user answers (l.7) "y", we log the entry (l.2).

\subsection{21-3 For each file on the system with a .bak suffix, where there either is no corresponding file without the .bak suffix, or the corresponding file is older than the .bak file, ask the user whether to rename the .bak file to remove the suffix. If the answer begins with Y or y, rename the file appropriately. You must handle file names containing single spaces correctly.}
\begin{verbatim}
#/bin/bash
exec 23<&0 24>&1 <<EOF
`find . -name "*.bak" -print`
EOF
while read BAK; do
    FILE=$(echo "$BAK" | sed 's|\.bak$||g')
    if [ ! -e "$FILE" ] || [ "${BAK}" -nt "${FILE}" ]; then
       echo -n "$FILE not found or older than $BAK. Replace $FILE? Y/N"
       read ans <&23
       if [ "$ans" = "y" ] || [ "$ans" = "Y" ]; then
	  mv -T "$BAK" "$FILE"
       fi
    fi
done
\end{verbatim}

\end{document}

\input{../report.input.tex}
\chapter{SCT}
\subsection{1-1}
 \verb=$@= looks at each argument individually, whereas \verb=@*= regards them as a collection as shown by the example below:

\begin{verbatim}
#foo.sh
#!/bin/bash
printf "%s\n" "$@"
printf "%s\n" "$*"
#EOF

 $ ./foo.sh meep moop
  meep
  moop
  meep moop
\end{verbatim}

\subsection{2-1}
\begin{verbatim}$@\end{verbatim} will make another evaluation of the parameter which can yield unwanted results with reserved characters. For instance, if fed \verb=`ls -F`= when an executable is present in the path (annotated in the output with \verb='*'=, e.g. \verb=foo*= ), it will evaluate the expression (\verb=foo*= becomes \em"all files beginning with\em \verb="foo"=) which might result in duplicate arguments.

\subsection{2-2}
Yes, and no. \verb=$*= could work, but we would still run into the issues mentioned in \emph{2-1} with evaluation of the arguments provided. \verb=$*= would not work however, unless it is only one file provided as argument.

\subsection{3-1}
\begin{verbatim}
#/bin/bash
i=10
while [ $i -ne 0 ]
do
    echo "$i"
    i=$(($i - 1))
done
\end{verbatim}

\subsection{5-1}
Evaluates \verb=$line=, string position 0, 1 character(s).

\subsection{5-2}
: == true, which is needed as conditional blocks can't be empty.

\subsection{5-3}
\verb=$(command)=

\subsection{5-4}
Arithmetic expansion. Allows evaluation of arithmetic expressions and substitution of the result, allowing nesting for instance.

\subsection{5-5}

\verb=$ ls -a | grep "^\." | wc -l=

\subsection{6-1}
a) A backslash is used as en escape character in bash, and in this case it escapes the newline, allowing the script to be written on several (two here) lines while being evaluated as if it was a single line.

b) Earlier, we saw that command scope within () are executed in a new sub-shell. If replaced by \{\} the scope is executed in the current shell context.

\subsection{6-2}
\verb=local= creates a temporary object that is local to the current call frame, in this case a variable \verb=$mailserver=.

\subsection{6-3}
To demonstrate \verb=sleep=?

\subsection{7-1, 8-1}

See separate files.
\end{document}

\chapter{APT}
$\text{Errata: "[...]dpkg, is one of the most powerful"}^{\text{citation needed}}$
\subsection{1-1}
a) \verb=# apt-get install <package>=

b) \verb=# apt-get remove|purge <package>=

c) \verb=update= gets the latest index files from the repositories.

d) It upgrades all explicitly installed packages and (hopefully) their dependencies.

\subsection{1-2}
a) Prints a list of installed packages, unless a search pattern is given, in which case not installed in the indexes also are included (well, the ones that matches the search pattern).

b) Removes the package and its config files.

c) Lists files installed by \verb=bind9=.


% 2-1
% # /etc/apt/sources.list
% deb http://ftp.se.debian.org/debian squeeze
% deb http://ftp.se.debian.org/debian squeeze updates
%
% deb-src <url> <dist> <component>

% 3-1,2
% # apt-get install aptitude hping

% 4-1
% # apt-get remove hping

\subsection{6}
1)

2)

3)

4)

5) The package has broken dependencies.


%%% Doc Props
\documentclass[11pt,a4paper]{report}
\usepackage[utf8]{inputenc}
\usepackage[english]{babel}

% Packages
\usepackage{
  amsmath,
  titlesec,
  tikz,
  pifont,
  textcomp,
  fixltx2e,
  xcolor,
  fullpage,
  graphicx,
  afterpage,
  lastpage,
  float,
  fancyhdr,
  fancyvrb,
  parskip,
  xfrac,
  gensymb,
  framed,
  caption,
  subcaption,
  mathtools,
  pdfpages,
  etoolbox,
  enumitem,
  hyperref
}
\usepackage[nottoc,section,numbib]{tocbibind}
%\usepackage[europeancurrents,europeanvoltages,europeanresistors,europeaninductors,smartlabels]{circuitikz}
%\usepackage{pgfplotstable}
%\pgfplotsset{compat=1.12}
\usetikzlibrary{calc,matrix,positioning,arrows,shapes,trees,plotmarks,decorations.markings,shapes.geometric}
\usepackage[font={small,it}]{caption} %% Italics in captions
\tikzstyle{vecArrow} = [thick, decoration={markings,mark=at position 1 with {\arrow[semithick, fill=green]{triangle 60}}}, double distance=1.4pt, shorten >= 5.5pt, preaction = {decorate}, postaction = {draw,line width=1.4pt, green,shorten >= 4.5pt}]
\tikzstyle{innerWhite} = [semithick, green,line width=1.4pt, shorten >= 4.5pt]

\tikzset{
  treenode/.style = {align=center, inner sep=0pt, text centered,font=\sffamily},
  arnn/.style = {treenode, circle, white, font=\sffamily\bfseries, draw=black,fill=black, text width=1.5em},
  arnr/.style = {treenode, circle, red, draw=red, text width=1.5em, very thick},
  arnx/.style = {treenode, rectangle, draw=black,  minimum width=0.5em, minimum height=0.5em}
}

% PDF props
\hypersetup {
  bookmarks=true,              % show bookmarks bar?
  unicode=true,                % non-Latin characters in Acrobat’s bookmarks
  pdftoolbar=true,             % show Acrobat’s toolbar?
  pdfmenubar=true,             % show Acrobat’s menu?
  pdffitwindow=false,          % window fit to page when opened
  pdfstartview={FitH},         % fits the width of the page to the window
  pdftitle={TDDI41},                 % title
  pdfauthor={Oscar Petersson}, % author
  pdfsubject={},          % subject of the document
  pdfkeywords={}          % list of keywords
  pdfnewwindow=true,           % links in new window
  hidelinks,                   % hide links (removing color and border)
  linktoc=all,                 % parts of TOC made into links
}
\urlstyle{same}

% More Props
\graphicspath{{./fig/}}

% Comfy Command Redefinitions

%% First Page Information
\author{Oscar Petersson, Matteus Laurent\\oscpe262, matla782}
\title{TDDI41 Lab report}
\date{\today}
\newcommand{\course}{System Administration}
\newcommand{\coursenumber}{TDDI41, Linköpings universitet}
\newcommand{\programme}{Högskoleingenjörsutbildning i datateknik, 180 hp}
\newcommand{\examiner}{Jon Dybeck}
\newcommand{\institution}{IDA}
\newcommand{\reptype}{Laboration report}

%%% Document %%%
\begin{document}

\setcounter{secnumdepth}{-1}
\setcounter{tocdepth}{-1}

% Custom titlepage, empty backside, do not use with \maketitle
\pagestyle{empty}
  \makeatletter\begin{center}~\vfill
  \Large{\@author}\vskip .3cm
  \textbf{\LARGE{\@title}}\vskip .2cm
  \large{\programme}
  \vfill
\end{center}
\reptype{} - \@date\hfill Assistant:\\
\textbf{\course}\hfill\examiner\\
\coursenumber\hfill\institution
\makeatother
\newpage
\pagestyle{plain}
%% Table of Contents
%\addtocontents{toc}{\protect\hypertarget{toc}{}}
%\tableofcontents\label{sec:toc}
%\addtocontents{toc}{\protect\thispagestyle{empty}}
%\cleardoublepage
% Call First File Branch
%\input{./tex/main.tex}
%\input{./tex/LXB.tex}

%\end{document}

\chapter{LXB}
\section{Exercise 3: Navigating the Unix manual}
\subsection{3-1 Answer the following questions concerning sections of the Unix Manual. Most of the information you need can be found in the man page for the man command (you will need to think a little too).}
\subsubsection{a) Which are the nine sections of the Unix manual.}
\begin{tabular}{ll}
1&   Executable programs or shell commands\\
2&   System calls (functions provided by the kernel)\\
3&   Library calls (functions within program libraries)\\
4&   Special files (usually found in /dev)\\
5&   File formats and conventions eg /etc/passwd\\
6&   Games\\
7&   Miscellaneous (including macro packages and conventions), e.g. man(7), groff(7)\\
8&   System administration commands (usually only for root)\\
9&   Kernel routines
\end{tabular}

\subsubsection{b) Which section of the manual contains user commands such as cat and ls.}
Section 1

\subsubsection{c) Which section documents file formats, such as configuration files.}
Section 5

\subsubsection{d) Which section contains system administration commands, such as halt.}
Section 8
\subsection{4-1 The SYNOPSIS section briefly lists how a command is invoked. Read the man page for man and explain what the following command synopsis texts mean. Explain how each command may or must be invoked; you don't need to know what the commands actually do.}
a) \verb=mkpasswd PASSWORD SALT=
\begin{verbatim}command argument argument\end{verbatim}

b) \verb=uniq [OPTION]... [INPUT [OUTPUT]]=

\begin{verbatim}command [runtime options goes here]...
[optional input stream[optional output stream]]\end{verbatim}


c) \verb=gzip [-acdfhlLnNrtvV19 ] [-S suffix] [ name ... ]=

\begin{verbatim}
command [-available runtime flags w/o further arguments]
[-S flag needs argument (filename suffix)] [optional output name]
\end{verbatim}

d)

\begin{verbatim}chcon [OPTION]... CONTEXT FILE...
chcon [OPTION]... [-u USER] [-r ROLE] FILE...
chcon [OPTION]... --reference=RFILE FILE...\end{verbatim}

Three possible invokation syntaxes:
\begin{verbatim}
     command [runtime option] argument argument
     command [runtime option] [-u-flag argument] [-r-flag argument] argument
     command [runtime option] --reference-flag=argument argument
\end{verbatim}


\subsection{4-2 Read the man page for man, as well as some other man pages (e.g. for ls, uniq, chmod, and adduser) and answer the following questions.}
\subsubsection{a) What do you usually find in the DESCRIPTION section.}
A brief briefing of what the entry is about, i.e. for an executable - what the executable' function is, and a description of flags etc.

\subsubsection{b) Which section(s) usually document, in detail, what each command-line option does.}
Section 1

\subsubsection{c) Let's say you're reading the man page for a command and the information you are looking for isn't there. In which part can you find references to other man pages that might contain what you are looking for (have a look at the man page for reboot, imagining that you are trying to find a command that will turn the computer off, for an example).}
Under \textbf{SEE ALSO}

\subsubsection{d) In which section do you find information about which configuration files a program uses?}
Section 1, usually a subsection


\subsection{4-3 Use the man page for man to answer the following questions.}

\subsubsection{a) Sometimes there are several man pages (located in different sections) for the same keyword. Which command-line option to man can you use to display all of them.}
\verb=man -a <search term>=

\subsubsection{b) Sometimes you don't know the name of a command you are looking for, but can guess at a keyword related to the command. Which command-line option can you use to have man list all pages related to a specific keyword.}
\verb=man -k <keyword>=

\subsection{4-4 Display the man page for the ls command.}
\subsubsection{a) What does the ls command do.}
It lists info about files in a given search path.

\subsubsection{b) What option to ls shows information about file sizes, owner, group, permissions and so forth.}
\verb=-l=

\subsubsection{c) What does the -R option to ls do? (Don't forget to try it.) Report: Brief written answers to the questions above.}
Adds recursive directory search (lists subdirectories and their content as well).

\section{Exercise 5: Absolute and relative path names}
\subsection{5-1 In the example above name at least one relative path name indicating ssh if:}
\subsubsection{a) The current working directory is /usr/bin.}
\verb=./ssh=

\verb=../bin/ssh =

\subsubsection{b) The current working directory is /usr/local/bin.}
\verb=../../bin/ssh=

\section{Exercise 6: Long format chmod}
\subsection{6-1 It is possible to set individual permissions for user, group and others using chmod. Review the documentation and answer the following questions.}

\subsubsection{a) How can you set the permission string to user read/write, group read, others read using chmod in long format.}
\verb=# chmod a+rw-x,go-w <file>=

\subsubsection{b) How can you revoke group write permissions on a file without changing any other permissions.}
\verb=# chmod g-w <file>=

\subsubsection{c) How can you grant user and group execute permissions without changing any other permissions.}
\verb=# chmod ug+x foo=



\section{Exercise 7: Numeric file modes}
\subsection{7-1 What do the following numeric file modes represent.}

\subsubsection{a) 666}
\verb=rw-rw-rw-=

\subsubsection{b) 770}
\verb=rwxrwx---=

\subsubsection{c) 640}
\verb=rw-r-----=

\subsubsection{d) 444}
\verb=r--r--r--=

\subsection{7-2 What command-line argument to chmod allows you to alter the permissions of an entire directory tree.}
\verb=-R, --recursive=

\subsection{7-3 What does execute (x) permission mean on directories.}
Allows entering the directory and listing of its content.

\subsection{7-4 A user wants to set the permissions of a directory tree rooted in dir so that the user and group can list, read and write (but not execute) files, but nobody else has any access. Which of the following commands is most appropriate? Why?}

\subsubsection{a) chmod -R 660 dir}

\subsubsection{b) chmod -R 770 dir}

\subsubsection{c) chmod -R u+rw,g+rw,o-rwx dir}

\emph{(c)} is, arguably, the least wrong one, as it doesn't change possibly existing executive permissions for \verb=[ug]=. \emph{(a)} would remove any executive permissions on all files (and directories, making them inaccessible) for anyone, and \emph{(b)} would add executive permissions for \verb=[ug]= to all files and directories in the tree.

The prerequisits for \emph{(c)} would be that the executive permissions on the files already are correct.

The \emph{truly} correct answer would on the other hand be that a \emph{user} should not have the right to chmod recursively anywhere in the tree apart from their user-specific directories, and those should not be writable to anyone else.


\section{Exercise 8: Owner and group manipulation}
\subsection{8-1 How can you change the owner and group of an entire directory tree (a directory, its subdirectories and all the files they contain) with a single command.}
\verb=# chown user:group -R <path root>=

\section{Exercise 9: File manipulation commands}
\subsection{9-1 What does cd .. do.}
Changes current work directory to the parent directory.

\subsection{9-2 What does cd ../.. do.}
Changes current work directory to the parent directory of the parent directory.

\subsection{9-3 What information about a file is shown by ls -laF?}
Current directory content in the format:\\
\verb=filemode <> user group filesize(bytes) modification_time filename[entry type indicator]=

\subsection{9-4 In the following output from}
\begin{verbatim}
  ls -laFd dir dsp:
  drwxr-xr-x 22 dave staff 4096   Jan 12 2001 dir/
  crw-rw----  1 root audio 14, 3  Jan 22 2001 dspp
\end{verbatim}
\subsection{Explain the following:}
\subsubsection{a) What does the "c" at the beginning of the second line mean.}
Character special file, e.g. pipes, ports, devices etc ...

\subsubsection{b) What do "dave" and "staff" mean on the first line, and "root" and "audio" on the second.}
User permissions for the files are set to the users "dave/root" and groups "root/audio" respectively.

\subsubsection{c) Which users may write to the file dspp.}
The user "root" and any users in the group "audio".

\subsection{9-5 If you have two files, a and b, and you issue the command mv a b, what happens? Is there an option to mv that will issue a warning in this situation.}
rename file 'a' to 'b'. \verb=mv -i= will interactively ask for permission to overwrite.

\subsection{9-6 What option(s) to cp allows you to copy the contents of /dir1 to /dir2, preserving modification times, ownership and permissions of all files.}
\verb=$ cp -a /dir1/*.* /dir2/=

\subsection{9-7 How do you make the file secret readable and writable by root, readable by the group wheel and completely inaccessible to everybody else.}
\verb=$ chown root:wheel secret && chmod 640 secret=

\section{Exercise 10: Shell init files}
\subsection{10-2 What init files does your shell use, and when are they used? (Hint: your shell has a man page, and somewhere near the end there is a section that lists which files it uses).}

\begin{tabular}{ll}
\verb=/bin/bash= &
	The bash executable\\
\verb=/etc/profile= &
	The systemwide initialization file, executed for login shells\\
\verb=/etc/bash.bashrc= &
	The systemwide per-interactive-shell startup file\\
\verb=/etc/bash.bash.logout= &
	The systemwide login shell cleanup file, executed when a login shell exits\\
\verb=~/.bash_profile= &
	The personal initialization file, executed for login shells\\
\verb=~/.bashrc= &
	The individual per-interactive-shell startup file\\
\verb=~/.bash_logout= &
	The individual login shell cleanup file, executed when a login shell exits\\
\verb=~/.inputrc=&
	Individual readline initialization file
\end{tabular}

\section{Exercise 11: Manipulating environment variables}
\subsection{11-1 Use the env command to display all environment variables. What is PATH set to (you might want to use grep to find it)? What is this variable used for (the man pages for your shell might be helpful in answering this question).}
\verb+PATH=/bin:/sbin:/usr/local/sbin:/usr/local/bin:/usr/bin:/usr/bin/site_perl:\+\\
\verb+/usr/bin/vendor_perl:/usr/bin/core_perl+

\verb=$PATH= is a list of paths from where the shell will recognize executables. Executables in \verb=$PATH= can be executed from any search path without writing the full path to the executable.

\subsection{11-2 Use echo to display the value of HOME. What does the HOME variable normally contain.}
\verb=$HOME= is the user "default" directory. Here the users document, user-specific configurations etc. are stored.

\subsection{11-3 Prepend /data/kurs/adit/bin:/data/kurs/TDDI09/bin to the variable PATH. The easiest way to accomplish this is to use variable expansion in the right-hand side of the assignment.}
\begin{verbatim}$ PATH='/data/kurs/adit/bin:/data/kurs/TDDI41/bin:'${PATH}\end{verbatim}

\section{Exercise 12: Redirecting output}
\subsection{12-1 Where will stdout and stderr be redirected in the following examples?}
\verb=a) command >file1 b) command 2>&1 >file1 c) command >file1 2>&1=
\subsection{When answering these, remember that the order of redirections matters.}
\begin{tabular}{l|ll||l|ll}
stdout: &&&	stderr:&&\\\hline
 &  file1 &console&	  & file1& console\\\hline
a&  y&	 n &	  a&  n&     y\\
b&  y&   n &		b&  n&     y\\
c&  y&	 n &		c&  y&	   n
\end{tabular}

\section{Exercise 13: Pipelines}
\subsection{13-1 What do the following commands do?}
\begin{verbatim}
a) ls | grep -i doc
b) command 2>&1 | grep -i fail
c) command 2>&1 >/dev/null | grep -i fail
\end{verbatim}
a) List directory content, select only the lines which contains the ordered character set "doc" ignoring upper/lower case.

b) Redirects stderr output of \verb=command= to stdout which is then parsed for "fail" (ignoring u/l case)

c) As b, but will also put the stdout output into The Great Void.
\subsection{13-2 Write command lines to perform the following tasks.}
\subsubsection{a) Output a recursive listing (using ls) of your home directory, including invisible files, to the file /tmp/HOMEFILES.}
\verb=$ ls -aR ~ > /tmp/HOMEFILES=

\subsubsection{b) Find any files (using find) on the system that are world-writable (i.e. the write permission for "others" is set). Error messages should be discarded (redirected to /dev/null). This command is actually useful for auditing the security of a system - world-writable files can be security risks.}
\verb?# find / -perm -o=w -type f 2>/dev/null?


\subsection{Exercise 14: Processes and jobs}
\subsection{14-1 Create a long running process by typing ping 127.0.0.1. Suspend it with vZ and bring it to the foreground with fg. Terminate it with vC.}

\subsection{14-2 Create a long running process in the background by typing} \verb=ping 127.0.0.1 >/dev/null&=.
\subsection{Find out its process id using ps and kill it using kill.}
\begin{verbatim}
$ ping 127.0.0.1 >/dev/null&
kill $(ps -Ao pid=,comm --no-headers | grep ping | grep -o -E '[0-9]+')
\end{verbatim}

(Or, just "ps", look for ping, send \verb=kill <found PID>=), but that's a rather boring way, isn't it? ;)

\subsection{14-3 What does the command kill -9 pid do, where pid is the number of a process? What does kill -9 -1 do? Read the documentation to figure the last one out as it is a somewhat dangerous command.}
\verb=$ kill -9 <pid>=

This line sends \textbf{SIGKILL} to the process with given PID. This can not be caught or ignored by the process.

\verb=$ kill -9 -1=

This line sends \textbf{SIGKILL} to all processes with a PID larger than 1, i.e. every process apart from the init process started by the boot manager.


\subsection{14-4 Create a long running process in the background. Kill it using pkill. The pkill command is very useful when you need to kill several processes that share some attribute (such as a command name).}
\verb=$ pkill ping=


\section{Exercise 16: Using the pager less}
\subsection{16-1 What keystroke in less moves to the beginning of the file.}
\subsection{16-2 What keystroke in less moves to the end of the file.}
\subsection{16-3 What would you type in less to start searching for "option".}
\subsection{16-4 What would you type in less to move to the next match for "option".}
BOF: g\\
EOF: G\\
forward search: /option\\
backward search: ?option

\subsection{16-5 Locate the package documentation for the ssh package and answer the following questions by reading the README.Debian.gz file (hint: remembering the answers to these questions may be useful in the project).}

\subsubsection{a) What is the default setting for ForwardX11.}
For security reasons, \verb=X11Forwarding= is disabled by default, this goes for Debian's \verb=openssh-server= package as well.

\subsection{b) If you want X11 forwarding to work, what other package(s) need to be installed on the server.}
\verb=xauth= at a bare minimum, plus of course any related packages for whatever functionality desired.



\subsection{Exercise 17: Using non-interactive text editors (this exercise is optional)}
\subsection{17-1 Use sed to change all occurrences of "/bin/tcsh" to "/bin/sh" in /etc/passwd (output to a different file). This exercise is optional.}
\verb=# sed -i "s|bin/tcsh|bin/sh|g" < /etc/passwd > foo=

\subsection{17-2 Examine the files shadow and passwd in the directory /data/kurs/TDDI09/labs/lxb. Use paste and awk to output a file where each line consists of column one from passwd and column two from the corresponding line in shadow. The printf function in awk is helpful here. This exercise is optional since it goes beyond the basics.}
\begin{verbatim}$ paste <(awk -F ":" '{print $2}' passwd ) <(awk -F ":" '{print $3}' shadow)\end{verbatim}

\section{Exercise 18: Log files}
\subsection{18-1 What does tail -f /var/log/syslog do.}
Prints the last 10 lines of \verb=/var/log/syslog= and keeps parsing.

\subsection{18-2 If you want to extract the last ten lines in /var/log/syslog that are related to the service cron, what command would you use? (Hint: the grep command can search for matching lines in a file).}
\verb=$ cat /var/log/syslog | grep cron=

%\section{Part 5 Errata:}
% The default configuration for most contemporary distributions is systemd, regardless of what one thinks of that. Sure, sysvinit might be a reasonable (at least you can reason wether it is or not ...) choice regarding this course, but that does not change the fact that the major distributions likely to be found in a professional environment (Debian, Fedora, SUSE, Red Hat, \*buntu, Mint)- except for legacy systems - are not running sysvinit.
% Errata-errata: Well, to be fair, likely to be found is old legacy systems, and Ubuntu is running upstart etc. So, never mind this crap.

\section{Exercise 19: Booting Linux}
\subsection{19-1 What services are started when your system boots into run level 2 (i.e. not only services exclusive to run level 2, but all services started as the computer boots into run level 2).}
bootlogs, syslog, ssh, cron, rmnologin, stopbootlogs

\subsection{19-2 If you wanted to restart the ssh process, what command would you use.}
\begin{verbatim}
$ systemctl restart sshd.service
-bash: systemctl: command not found
\end{verbatim}

Oh right, old habits die hard ... :p

\verb=$ /etc/init.d/ssh restart=


\section{Exercise 20: Mostly hard stuff}
\subsection{20-1 Where will stdout and stderr be redirected in the following examples:}
\verb=a) command 2>&1 >file1 b) command >file1 2>&1=
\emph{see 12-1 b/c}

\subsection{20-2 Using only ps, grep and kill, write a command that kills any process with "linux" in the name. Note that you must avoid killing the grep process itself (this can be done by crafting the regexp appropriately)! This command line can be used to rapidly terminate any UML running on the system.}
\begin{verbatim}$ kill $(ps -Ao pid,comm --no-headers | grep ping | grep -o -E '[0-9]+')\end{verbatim}

\subsection{20-3 Use your shell's alias or function features to create a shell command named gzless that decompresses a gzip-compressed file and pipes it into less.}
\begin{verbatim}
# ~/.aliases
gzless() {
  gunzip -c $1 | less
}
\end{verbatim}

\subsection{20-4 I/O redirection in the shell is somewhat limited. In particular, you can't handle programs that write directly to the terminal device (i.e. use a pseudo-terminal) rather than use the file descriptors. For situations like that, there's socat. Since you're an advanced user, you really should learn about socat.}
\subsubsection{a) Use socat so you can log in and execute commands on a host non-interactively using the password authentication method. Essentially, you should be able to do the equivalent of}
\verb= cat commandfile | ssh user@remote.host =
\subsubsection{, where commandfile contains the stuff you would normally type interactively.}
\begin{verbatim}
$ file=$(cat commandfile) && \
  (sleep 3; echo PASSWD; sleep 1; echo "$file"; sleep 1) |\
  socat - EXEC:'ssh -l USER remote.ip.addr',pty,setsid,ctty
\end{verbatim}

\subsubsection{b) Do something else with socat that's neat. Come up with something on your own, perhaps using the network.}

\subsection{20-5 Write a command that renames a bunch of files to be all lowercase (e.g. BOFH.GIF is renamed to bofh.gif). You do not have to worry about spaces (but you should, just on general principle). The tr command may be useful here.}
\begin{verbatim}
#!/bin/bash

for OLD in `find $1 -depth`
do
   NEW=`dirname "${OLD}"`/`basename "${OLD}" | tr '[A-Z ]' '[a-z ]'`
   if [ "${OLD}" != "${NEW}" ]
   then
	[ ! -e "${NEW}" ] && mv -T "${OLD}" "${NEW}" \
	|| echo "Could not rename ${OLD}. ${NEW} already exists."
   fi
done
\end{verbatim}

\subsection{20-6 Examine the files /data/kurs/TDDI09/labs/lxb/passwd and shadow (same directory). Use paste and awk to output a file where each line consists of column one from passwd and column two from the corresponding line in shadow. The printf function in awk may be helpful here.}
\emph{see 17-2}

\subsection{20-7 Sometimes you need to change the IP address of a computer. Assuming that all the files that need to be changed are somewhere in /etc (or a subdirectory thereof), what command will list all relevant files and not print any error messages.}
\verb=# find ./ -type f | xargs grep "$oldip" 2>/dev/null=
...where \verb=$oldip= is the IP-address we are looking for, of course.

\subsection{20-8 Write a command line that changes the IP address stored in the variable (shell or environment) OLD to the IP address stored in the variable NEW in all (regular) files in /etc or its subdirectories. The commands find, xargs and sed may be useful here.}
\verb=# find /etc -type f -exec sed -i "s|$OLD|$NEW|g" {} \;=

\subsection{20-9 Write a command that uses ssh to log in to all computers w0001 through w2000 and runs the w command on each one. You might want to read about arithmetic substitution in bash for this one. How could you avoid having ssh ask for a password for each computer without setting an empty password or resorting to host-based authentication.}

Avoiding password login would be done through pubkey-based access:

\begin{verbatim}
#/etc/ssh/sshd_config
 PasswordAuthentication no
 PermitEmptyPasswords no
 [...]
#EOF
\end{verbatim}

Generating and importing keys ...
\verb=ssh-keygen=

We could use \verb=sshpass -p <PASSWD> ssh-copy-id -i ~/.ssh/<id-file> <user>@<host>= to avoid manually entering the passwords, at least if they are identical on all hosts...

Once sorted (phew) we'll simply run:

\begin{verbatim}
$ for client in `seq 0001 2000`
  do
    ssh user@$client <<FOOXECUTETHIS
    w
    FOOXECUTETHIS
  done
\end{verbatim}
\section{Exercise 21: Quite hard stuff (optional)}
\subsection{21-1 Write a command that lists, for the last ten unique users to log in to the system, the last time they logged in using ssh (users can be found using last, and ssh logins in /var/log/auth.log, which is typically only readable by root).}

\subsection{21-2 Explain the following fairly contrived code, in particular all the I/O redirections.}
The code starts out with directing two file descriptors, one for input and one for output (l.2). We look for .bak-files throughout the file system and print their path on a line each (l.3). We read these lines (l.5), ask (l.6) the user to confirm that we should log the entry to \verb=/tmp/RECORD (l.2)=. If (l.8) the user answers (l.7) "y", we log the entry (l.2).

\subsection{21-3 For each file on the system with a .bak suffix, where there either is no corresponding file without the .bak suffix, or the corresponding file is older than the .bak file, ask the user whether to rename the .bak file to remove the suffix. If the answer begins with Y or y, rename the file appropriately. You must handle file names containing single spaces correctly.}
\begin{verbatim}
#/bin/bash
exec 23<&0 24>&1 <<EOF
`find . -name "*.bak" -print`
EOF
while read BAK; do
    FILE=$(echo "$BAK" | sed 's|\.bak$||g')
    if [ ! -e "$FILE" ] || [ "${BAK}" -nt "${FILE}" ]; then
       echo -n "$FILE not found or older than $BAK. Replace $FILE? Y/N"
       read ans <&23
       if [ "$ans" = "y" ] || [ "$ans" = "Y" ]; then
	  mv -T "$BAK" "$FILE"
       fi
    fi
done
\end{verbatim}

\end{document}


%\end{document}

\subsection{Exercise 1: Very basic RAID theory}
\textbf{RAID} originally stood for \emph{"Redundant Array of Independent Disks"}, but
nowadays, it is commonly known as \emph{"Array of Independent Disks"}, i.e. a
collection of storage media working as a single logical unit.

\textbf{JBOD} (\emph{"Just a Bunch Of Disks"}) is multiple drives exposed as individual
devices. Sometimes, the conception of spanned disks is expressed as 'JBOD'.

\textbf{RAID-0} (\emph{"Striping"}) distributes the content (near-) equally across all disks to increase throughput. Any disk failure results in a full array failure.

\textbf{RAID-1} (\emph{"Mirroring"}) writes the data set identically to two drives, basically allowing a "backup" in case of one of the disks failuring.

\textbf{RAID-5} (\emph{"Block-level striping with distributed parity"}) requires at least three disks, distributing data across two with parity on the third. This allows one disk to fail with the possibility to recover the array.

\textbf{RAID-6} (\emph{"Block-level striping with double distributed parity"}) requires at least four disks, distributing data as RAID 5 but with parity on two disks.

\textbf{RAID 0+1} (\emph{"Mirrored stripes"}) stripes data across two sets of disks which are in turn mirrored.

\textbf{RAID 1+0} (\emph{"Stripe of mirrors"}) mirrors data and stripes the mirrored sets.

\textbf{Hot Spare} is a disk added to a failure redunant array, such as RAID-5, which is used only upon an array failure, at which point the array can start to rebuild instantly.

\subsection{Exercise 2: Volume management with LVM2}

A physical disk is divided into one or more \textbf{Physical volumes} (\verb=pv[command]=). Combining these makes  \textbf{Volume groups} (\verb=vg[command]=) which consists of any number of \textbf{Logical volumes} (\verb=lv[command]=).

\subsection{Exercise 3: File Systems}

The main differences between ext2 and ext3 is the addition of journaling (block and metadata) in ext3, and the possibility to grow a mounted file system on the latter.

Ext$n$ file systems are created with the command: \\
\verb=  mkfs.ext=$n$\verb= <device>=

Ext file systems are resized with:
\verb=  resize2fs <device> <size>=

\verb=fsck= is used to check and repair Linux file systems.

\subsection{Exercise 11: Some theory}
\subsubsection{11-1 Why is RAID 1+0 generally considered more reliable than RAID 0+1?}
Lets assume we have six drives, (1-6). For RAID 0+1 we use disk 1-3 striped, and mirror the stripe to disks 4-6. For RAID 1+0 we mirror disk 1 to disk 2, 3 to 4, and 5 to 6, and stripe the three sets.

Now, consider that we have two mirrored sets in RAID 0+1 and three in RAID 1+0. Should one disk in the 1-3 array fail in RAID 0+1 and one in the 4-6 during recovery, the array is doomed. However, for the RAID 1+0 array to fail, one disk in each set could fail and we'd still be able to recover the array, increasing the likelyhood of recovering the array before it is too late. So, even though the odds of two (contentwise) identical disks failing, the fault tolerance is increased.

\subsubsection{11-2 As individual disk sizes grow, RAID-5 is no longer considered useful, and RAID-6 is also rapidly becoming obsolete. Why?}
Mirrored setups, such as RAID 1+0, takes less time to recover. Parity schemes require all blocks of the drives in the array, whereas the the earlier only require those of the failed drive. This means that recovery of failed parity-based arrays are more likely to have a second (or third ...) failure during the recovery, rendering the array useless. All this assumes that read and write speeds does not follow the increase in disk sizes, but when did that happen last? ;)

\subsubsection{11-3 What is the point of the file system journal in e.g. ext3?}
One purpose is to reduce the amount of I/O operations by letting multiple operations be done in one go. Another purpose is to get a transaction behavoiur, is case of interruptions in the process.

\subsubsection{11-4 Explain the purpose of using a volume manager, like LVM}
We'd say there are two main purposes. The first is to allow disks to be added and replaced (hot swap implied) without downtime. The second is to allow snapshots on file systems which does not come with such functionality.

\subsubsection{11-5 Optional: compute the probability of data loss over three years in a RAID-6 array consisting of 24 6TB SAS disks, with two hot spares (i.e. 22 disks in the RAID set). The RAID controller has 150MB/s throughput of which 80\% can be used to rebuild. Assume MTBF of 145000 hours, latent sector error of 10E-13.}
We'll get back to that if there is time ...
% Report: Answers to the questions above.

\end{document}
