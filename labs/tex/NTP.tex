\input{../report.input.tex}
\chapter{NTP}
\subsection{3-3}
% 3-3 Explain the output of ntpq -p.
\begin{verbatim}
# ntpq -p
     remote           refid      st t when poll reach   delay   offset  jitter
==============================================================================
*ida-gw.sysinst. 192.36.143.130   2 u    3   64   37  130.222   55.492  64.905
 130.236.178.159 .BCST.          16 u    -   64    0    0.000    0.000   0.031
\end{verbatim}
\vspace{1cm}
\begin{tabular}{l|l}
  * & current time source \\
  remote & target of sync \\
  refid & remote source's sync source \\
  st & stratum level of source\\
  t & types available (u = unicast)\\
  when & time since last received packet\\
  poll & poll interval (log2 seconds)\\
  reach & Octal binary history of packets\\
  delay & roundtrip delay\\
  offset & server time difference (ms)\\
  jitter &difference in milliseconds between two samples
\end{tabular}

\end{document}
