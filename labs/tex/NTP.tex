\input{../report.input.tex}
\chapter{NTP}
\subsection{Exercise 2: The Network Time Protocol}
\subsubsection{2-1 Explain the concept of "stratum" in a clock hierarchy (e.g. NTP, but also synchronized networks and other clock hierarchies).Why is it necessary to have a hierarchy of clocks rather than just a bunch of reference clocks.}
Stratum 0 are reference clocks, usually atom clocks. An NTP server which gets its time from those are labeled stratum 1, and the servers which sync to stratum 1 are labeled stratum 2, and so on. Stratum 16 is an unsynchronized device.
Stratum tiers are necessary for backup and "sanity checks".

\subsubsection{2-2 How large a difference between the reference clock and the system clock does NTP accept and attempt to adjust.}
Less than 1000 s, but greater than 128 ms.

\subsubsection{2-3 NTP usually works by speeding up or slowing down the clock, not setting it outright. Why.}
Setting the time is a security risk and a reliability risk with time stamps occuring twice or not at all.

\subsection{Exercise 3: Install and configure NTP server and clients}
\subsubsection{3-3 Explain the output of ntpq -p.}
\begin{verbatim}
# ntpq -p
     remote           refid      st t when poll reach   delay   offset  jitter
==============================================================================
*ida-gw.sysinst. 192.36.143.130   2 u    3   64   37  130.222   55.492  64.905
 130.236.178.159 .BCST.          16 u    -   64    0    0.000    0.000   0.031
\end{verbatim}
\vspace{1cm}
\begin{tabular}{l|l}
  * & current time source \\
  remote & target of sync \\
  refid & remote source's sync source \\
  st & stratum level of source\\
  t & types available (u = unicast)\\
  when & time since last received packet\\
  poll & poll interval (log2 seconds)\\
  reach & Octal binary history of packets\\
  delay & roundtrip delay\\
  offset & server time difference (ms)\\
  jitter &difference in milliseconds between two samples
\end{tabular}

\end{document}
