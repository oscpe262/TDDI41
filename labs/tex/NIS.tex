\input{../report.input.tex}
\chapter{NIS}
\section{Exercise 1: Review and preparation}
\subsection{1-1 Read about NIS and answer the following questions.}
\subsubsection{a) What is a NIS map?}
A NIS map is essentially a map structure with a key and a corresponding value.

\subsubsection{b) What is a NIS domain? What is the NIS domain of your server and clients.}
A network of a master server, optional slave servers, and a set of clients. For us, the B4-server will act as NIS server and the rest will be clients.

\subsubsection{c) What is the name of the map that is used to look up users by name.}
netgroup

\subsubsection{d) How do users change passwords when NIS is in use? How does that work.}
Preferrably, they use \verb=rpasswd= which changes the password over a SSL connection.

\subsection{1-2 Read the documentation for the commands ypcat, ypwhich and ypbind and answer the following questions.}
\subsubsection{a) What is ypcat used for.}
Print values of all keys from the NIS database specified by mapname.

\subsubsection{b) What is ypwhich used for.}
Returns the NIS server name.

\subsubsection{c) What is ypbind used for.}
Maintaining NIS binding information.

\subsubsection{d) What does the -k command-line option to ypcat do.}
Displays map keys.

\subsubsection{e) If you want to list all users on a NIS client, what command would you use.}
\verb=ypcat netgroup= %? passwd?

\subsubsection{f) If you want to see which NIS server a client is bound to, which command would you use.}
\verb=ypwhich [hostname]= (empty for localhost)

\subsubsection{g) If you want to see the NIS domain of a client, what command would you use.}
\verb=ypdomainname= %?

%Report: No report required.

\subsection{Exercise 2: Directory services}
\subsubsection{2-1 Why use a directory service?}
Why not? It's quite handy ...

\subsubsection{ Name one alternative to a directory service.}
No directory service.

\subsubsection{2-2 Name at least two common directory services other than NIS.}
Microsoft Active Directory, LDAP

%Report: Answers to the questions above.
\end{document}
