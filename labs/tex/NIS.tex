\input{../report.input.tex}
\chapter{NIS}
\section{Exercise 1: Review and preparation}
\subsection{1-1 Read about NIS and answer the following questions.}
\subsubsection{a) What is a NIS map?}
A NIS map is essentially a map structure with a key and a corresponding value.

\subsubsection{b) What is a NIS domain? What is the NIS domain of your server and clients.}
A network of a master server, optional slave servers, and a set of clients. For us, the B4-server will act as NIS server and the rest will be clients.

\subsubsection{c) What is the name of the map that is used to look up users by name.}
netgroup

\subsubsection{d) How do users change passwords when NIS is in use? How does that work.}
Preferrably, they use \verb=rpasswd= which changes the password over a SSL connection.

\subsection{1-2 Read the documentation for the commands ypcat, ypwhich and ypbind and answer the following questions.}
\subsubsection{a) What is ypcat used for.}
Print values of all keys from the NIS database specified by mapname.

\subsubsection{b) What is ypwhich used for.}
Returns the NIS server name.

\subsubsection{c) What is ypbind used for.}
Maintaining NIS binding information.

\subsubsection{d) What does the -k command-line option to ypcat do.}
Displays map keys.

\subsubsection{e) If you want to list all users on a NIS client, what command would you use.}
\verb=ypcat netgroup= %? passwd?

\subsubsection{f) If you want to see which NIS server a client is bound to, which command would you use.}
\verb=ypwhich [hostname]= (empty for localhost)

\subsubsection{g) If you want to see the NIS domain of a client, what command would you use.}
\verb=ypdomainname= %?

%Report: No report required.

\subsection{Exercise 2: Directory services}
\subsubsection{2-1 Why use a directory service?}
Why not? It's quite handy ...

\subsubsection{ Name one alternative to a directory service.}
No directory service.

\subsubsection{2-2 Name at least two common directory services other than NIS.}
Microsoft Active Directory, LDAP

%Report: Answers to the questions above.

\subsection{Exercise 5: Configure the clients' name service switch}
\subsubsection{5-1 What is the difference between using "compat" and using "files nis" as the list of sources for e.g. passwords in the name service switch.}
Compat adds the possibility of special entries which enables partial (rows and/or columns) fetching of NIS maps. The local files remain the primary source, and the special entries are read from these.

%\subsubsection{5-2 Configure the name service switch on your clients so they use NIS for as much as possible. Note that you should still use local files as the first information source. Please do not use compat unless you intend to use the special features it provides.}

\subsubsection{5-3 Why should you use local files as the first information source instead of NIS.}
A strong argument for at least reading local login files, before querying the respective NIS map, is that if that map has a user entry for 'root' (a mistake), the opposite would effectively impose a network-global login for root.

However, even if a 'root' entry did not exist in the NIS map, using a "NIS first, local second" setup opens up for the same problem in case an attacker gains access to the NIS server, editing such an entry in.

%Report: Automated tests that show that the clients are now using NIS as expected. Answer to the question above.

%Note that your clients should get as much information as possible from NIS. The only information that should remain local is information that truly is local as well as information that never changes.


\end{document}
